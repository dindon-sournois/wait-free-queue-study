% *** Authors should verify (and, if needed, correct) their LaTeX system  ***
% *** with the testflow diagnostic prior to trusting their LaTeX platform ***
% *** with production work. The IEEE's font choices and paper sizes can   ***
% *** trigger bugs that do not appear when using other class files.       ***                          ***
% The testflow support page is at:
% http://www.michaelshell.org/tex/testflow/

\documentclass[10pt,journal,compsoc]{IEEEtran}

% If IEEEtran.cls has not been installed into the LaTeX system files,
% manually specify the path to it like:
% \documentclass[10pt,journal,compsoc]{../sty/IEEEtran}

% Some very useful LaTeX packages include:
% (uncomment the ones you want to load)

% \ifCLASSOPTIONcompsoc
  % IEEE Computer Society needs nocompress option
  % requires cite.sty v4.0 or later (November 2003)
  % \usepackage[nocompress]{cite}
% \else
  % normal IEEE
  % \usepackage{cite}
% \fi

% *** GRAPHICS RELATED PACKAGES ***
%
\ifCLASSINFOpdf
  % \usepackage[pdftex]{graphicx}
  % declare the path(s) where your graphic files are
  % \graphicspath{{../pdf/}{../jpeg/}}
  % and their extensions so you won't have to specify these with
  % every instance of \includegraphics
  % \DeclareGraphicsExtensions{.pdf,.jpeg,.png}
\else
  % or other class option (dvipsone, dvipdf, if not using dvips). graphicx
  % will default to the driver specified in the system graphics.cfg if no
  % driver is specified.
  % \usepackage[dvips]{graphicx}
  % declare the path(s) where your graphic files are
  % \graphicspath{{../eps/}}
  % and their extensions so you won't have to specify these with
  % every instance of \includegraphics
  % \DeclareGraphicsExtensions{.eps}
\fi
% graphicx was written by David Carlisle and Sebastian Rahtz. It is
% required if you want graphics, photos, etc. graphicx.sty is already
% installed on most LaTeX systems. The latest version and documentation
% can be obtained at: 
% http://www.ctan.org/pkg/graphicx
% Another good source of documentation is "Using Imported Graphics in
% LaTeX2e" by Keith Reckdahl which can be found at:
% http://www.ctan.org/pkg/epslatex
%
% latex, and pdflatex in dvi mode, support graphics in encapsulated
% postscript (.eps) format. pdflatex in pdf mode supports graphics
% in .pdf, .jpeg, .png and .mps (metapost) formats. Users should ensure
% that all non-photo figures use a vector format (.eps, .pdf, .mps) and
% not a bitmapped formats (.jpeg, .png). The IEEE frowns on bitmapped formats
% which can result in "jaggedy"/blurry rendering of lines and letters as
% well as large increases in file sizes.
%
% You can find documentation about the pdfTeX application at:
% http://www.tug.org/applications/pdftex

\hyphenation{op-tical net-works semi-conduc-tor}

\usepackage[backend=bibtex]{biblatex}
% \usepackage{csquotes}
\addbibresource{bibliography.bib}

\begin{document}
%
% paper title
% Titles are generally capitalized except for words such as a, an, and, as,
% at, but, by, for, in, nor, of, on, or, the, to and up, which are usually
% not capitalized unless they are the first or last word of the title.
% Linebreaks \\ can be used within to get better formatting as desired.
% Do not put math or special symbols in the title.

% Lecture d'article - Sythèse -
\title{Overview - ``A Wait-free Queue as Fast as Fetch-and-Add''}

\author{Loris~Lucido\\\small{\textit{Author:} Chaoran~Yang,
        John Mellor-Crummey}}% <-this % stops a space}

\IEEEtitleabstractindextext{%
% Note that keywords are not normally used for peerreview papers.
\begin{IEEEkeywords}
non-blocking queue, wait-free, fast-path-slow-path
\end{IEEEkeywords}}

\maketitle

\IEEEdisplaynontitleabstractindextext
\IEEEpeerreviewmaketitle

\newcommand{\para}[1]{\textbf{\textit{#1}}\hspace{3 mm}}

\section{Introduction}\label{sec:introduction}
Problematic:

we avoid the CAS retry problem by using Fetch and Fetch to find an index
can fast-path slow-path works for non-blocking algorithm using Fetch and Add ?


\section{Context and Related Work}\label{sec:context}
\para{MS-queue} The MS-queue \cite{Michael96simple} from is a classical simple
lock-free queue implemented with only compare-and-swap as primitive. As a
consequence, it is subject to the compare-and-swap retry problem. Even with a
low number of threads, performance are drastically reduced under contention
because of high probability of compare-and-swap failures.

MS-queue avoids the ABA problem by using double width compare-and-swap. Each
compare-and-swap to an address also increments a counter associated with that
address so that the comparison is done with one address and an integer
\cite{Herlihy08} \cite{Michael96simple}. \\

\para{Practical wait-free queue} One of the first implementation of a wait-free
queue with multiples enqueuers and dequeuers was designed . based on MS-Queue:
overhead of helping mechanism \\

\para{P-Sim queue} wait-free universal construction \\

\para{CC-queue} blocking, better performance, combining \\

\para{LCRQ} lock-free circular ring queue with Fetch and Add to get index, enqueue and
dequeue with double width compare and swap, which is not always available \\

\para{Fast-path slow-path} → with MS-queue as fast path


\section{Contribution}\label{sec:contribution}
The authors started from a basic obstruction-free queue similar to the base used
in the LCRQ algorithm. It is given in Listing \ref{lst:queue}. With this queue,
they constructed a wait-free queue by using the fast-path-slow-path methodology.

$FAA(x, v)$ atomically reads the value stored in the $x$ variable, increments it
by $v$ and returns the $x$ value pre-incrementation. $CAS(x, t, v)$ atomically
reads the value stored in $x$, compares it to $t$ and, if $x$ is equal to $t$
(success), replaces the value of $x$ by $v$. $CAS$ returns whether it has
successfully replaced the value or not.

\begin{lstlisting}[mathescape,
                   frame=single,
                   caption={An obstruction-free queue using an infinite array.},
                   label={lst:queue},
                   language=C]
Q: queue
T: pointer to tail
H: pointer to head
enqueue(x: var) {
  do t := FAA(&T, 1);
  while (!CAS(&Q[t], $\bot$, x));
}
dequeue(x: var) {
  do h := FAA(&H, 1);
  while (CAS(&Q[h], $\bot$, $\top$) and T > h);
  return (Q[h] == $\top$ ? EMPTY : Q[h]);
}
\end{lstlisting}

\para{Basic queue} This queue uses a shared infinite array (emulated) to store
elements. We will see later how this is done and how memory can be reclaimed.
Two special values are reserved : $\bot$ (bottom) and $\top$ (top). $\bot$
stands for empty cells as the queue is initially filled with $\bot$. $\top$
stands for unusable cells. When one thread dequeues an element, it marks the
cells with $\top$ to prevent other threads to enqueue an element in it.

To enqueue an element, one thread tries to find an available cell on the array,
a cell marked with $\bot$. One cannot enqueue an element in a $\top$ marked cell
because it could violate the FIFO property of the queue. To get a unique index
on the array, one thread uses fetch-and-add to increment the shared tail
reference. Considering the reference is shared by all threads, using
compare-and-swap instead of fetch-and-add would result in lots of failures.
After one thread is given a unique index, it needs to certify that the cell is
empty. It thus uses compared-and-swap to enqueue an element into the array. In
the event of compare-and-swap failure, the thread redoes the whole process until
the element is enqueued.

The dequeue operation works in a similar manner, one thread tries to find either a
cell filled with an element or marked with $\top$ by using fetch-and-add on a
shared head reference. If a thread stops on a $\top$ marked cell, the dequeue
returns \texttt{EMPTY}.

This queue is designed to be fast. Enqueue and dequeue may both fail if a
dequeuer thread marks the candidate cell for the enqueue unusable. As a result,
it is neither wait-free or lock-free because it is prone to livelocking. As only
one thread is guaranteed to progress, this queue is only obstruction-free.
\medskip

\begin{figure}
    \caption{The fast-path-slow-path methodology \cite{Kogan:2012:MCF:2370036.2145835}.}
    \label{fig:fpsp}
    \center
    \includegraphics[width=0.6\linewidth]{img/fpsp.pdf}
\end{figure}

\para{Fast-path-slow-path} To construct a wait-free realization of the previous
queue, the authors use the fast-path-slow-path methodology. As shown in figure
\ref{fig:fpsp}, one thread tries to enqueue or dequeue an element using a
fast-path, an algorithm similar to the previous queue. A maximum number of
failures is set. If a thread failed too many times to apply its operation, it
falls back on a slow-path.

For example, if a thread fails to enqueue an element, it publishes an enqueue
request. When other threads want to dequeue an element, they look at all pending
enqueue requests, and eventually help one request to complete. The enqueuer
thread then keeps trying to enqueue the element until it or another succeeds.

Threads are linked in a ring, they keep a reference to a \textit{peer} to which
they help the operation to complete. Each time a thread successfully helps
another thread, it updates his \textit{peer} to the next thread in the ring so
that each
thread happens to help every one at some point. \medskip

\para{Memory reclamation} The authors designed their queue with an infinite
array. To do so, the queue is split into segments as a linked list. Each segment
contains a fixed number of cells. The list of segments is expanded as necessary.
Because the tail and head pointers are never decremented, segments no longer in
use need to be freed. A segment can be freed when the head and tail pointers of
the queue move past the range of cells handled by this segment.

After one thread dequeues an element, the thread tries to reclaim memory from
unused segments. They use compare-and-swap to achieve mutual exclusion so that
if several threads try to reclaim memory, only one should succeed. Each thread
keeps a reference to its currently used segment. The \textit{cleaner} thread
ensures that no thread keeps a reference to a segment about to be freed and
changes it if needed. \medskip

\para{Wait-free guaranty} The number of tries on the obstruction-free fast-past
is bounded. After sufficient failures, the algorithm falls back on the
slow-path. Each thread helps every other thread at some point. If the operation
of one thread continuously fails to complete, it will definitely complete when
all other threads become his helper.


\section{Discussion}\label{sec:discussion}
Fetch and Add can fails (at hardware level)


\section{Conclusion}\label{sec:conclusion}
their design can serve as an example of how to create other wait-free object

wait-free at the cost of complexity


\printbibliography

\end{document}
