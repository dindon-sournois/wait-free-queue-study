Fast concurrent data structures are necessary to exploit the power of multi-core
processors and multi-processors machine. To deliver high scalable performance,
applications may need shared objects (data structures) with scalable operations.
Each operation of an object (adding, removing, modifying an element) much
perform \textit{reasonably} well even when the number of threads sharing an
object is high.

One way to construct concurrent data structures is by using an object with
blocking operations. An example of said object would be a concurrent queue whose
operations have been protected with mutual exclusion (locks). Unexpected delay
from one thread prevents the others to use the object. The source of the delay
is multiple : operating system preemption, cache miss, page fault, etc. Thus,
this approach is not scalable and it may lead to poor performance under high
contention, especially if the application uses more software threads than
hardware threads. That's why designing non-blocking objects is the topic of lots
of past and ongoing research. \\

\para{Wait-free object} We define here levels of \textit{process guarantees} on
each operation of a non-blocking object \cite{Yang:2016:WQF:3016078.2851168}.
Those are meant to predict how an object will behave under high contention. The
strongest guaranty is wait-free, which means an operation will complete in a
finite number of steps regardless of how many threads is accessing this object
and at what speed. Lock-free operations complete in a finite number of steps
only for \textit{some} threads. Obstruction-free is the weakest guaranty. Only
\textit{one} thread is guaranteed to complete an operation in a finite number of
steps.

While an object can be wait-free, this doesn't mean it will enable the
application to achieve high scalability. Actually, most past wait-free queue
implementations perform worse than others lock-free or blocking queue. To
achieve a wait-free guarantee, most wait-free implementations have to either
sacrifice parallelism or linearizability. \\

\para{Linearizability} Having multiple threads concurrently modifying a shared
object must not leave this object (a queue for example) in an
\textit{inconsistent} state. To ensure the designed object is correct, we need a
\textit{safety condition} for shared objects. The authors choose linearizability
for their design of concurrent queue. A sequence of operations is linearizable
if each operation appears to "take effect" instantaneously from the point of
view of other threads, regardless of when the operation started and returned.
The order of non-concurrent operations must also be preserved
\cite{Herlihy:1990:LCC:78969.78972}. The object is correct if every possible
sequences are linearizable.

An atomic operation or an operation protected by mutual exclusion is by
definition linearizable. When the operation is not atomic, one must consider the
point in time when changes made to the shared object become visible to the other
threads.

\para{Memory reclamation} Memory management is an important part of designing a
concurrent non-blocking algorithm. Garbage collection is not available in all
languages. Even it exists, one may need to do his own memory management to
be more efficient, especially if the garbage collection process is not lock-free
or wait-free.

The use of \texttt{malloc}, for example, may also violate the wait-free property
of the queue because \texttt{malloc} uses mutual exclusion to manage memory. One
solution is to have each thread manages a pool of pre-allocated cells
\cite{Herlihy08}. Enqueue operation take one cell from the pool, dequeue
operation put the cell back into the pool. While this solution avoids lock
contention, it is proned to the ABA problem which may corrupt memory.

The ABA problem occurs if one thread reads a shared address $A$, attempts to
replace the value of this cell with a compare-and-swap ($CAS$) and succeeds when
it should not. For example, if between the read and the comparison, another
thread replaced the cell $A$ to $B$, and then replaced $B$ to $A$ again, the
compare-and-swap may succeed under the assumption that the cell is still the
same than before.

One can also use other memory management implementations like \texttt{jemalloc},
\texttt{TCMalloc}, or \texttt{Lockless}. \\

\para{Consensus number} The use of atomic primitives such as compare-and-swap is
critical in designing Lock-free algorithm. To understand why, we must consider
the consensus problem. A consensus is reached between $n$ threads if those $n$
threads can \textit{decide} on a value : each thread proposes one value to the
others and then chooses one value. The consensus is reached if all threads
decide the same value (consistency requirement) and if the common decision is
one of the initially proposed value (validity requirement).

Compare-and-swap primitive has an infinite consensus number
\cite{Herlihy:1991:WS:114005.102808} : it can solve the consensus problem for an
infinite number of threads. Compare-and-swap holds another important property
which is universality. An object is universal for $n$ threads if and only if it
has a consensus number greater or equal to $n$. It is possible to implement any
concurrent object in a wait-free manner if and only if the operation system
provides an universal object. Thus, compare-and-swap can be used to implement
any arbitrary concurrent wait-free object for any number of threads. This
statement holds for other infinite consensus number primitive like
load-linked/store-conditional. \\

\para{Contribution} Unfortunately, the use of compare-and-swap on a variable
shared by several threads can drastically reduce performance even under low
contention (compare-and-swap retry problem
\cite{Morrison:2013:FCQ:2517327.2442527}). In this paper, C. Yang and J.
Mellor-Crummey expose a new queue that is wait-free, linearizable, and delivers
high performance.

They avoid common issues with past designs such as the compare-and-swap problem
by also relying on a fetch-and-add primitive ($FAA$). Fetch-and-add has
consensus number two but is less expensive than compare-and-swap and cannot
fail. They designed their wait-free queue using a methodology called
fast-path-slow-path \cite{Kogan:2012:MCF:2370036.2145835}. It can transform
non-blocking lock-free queue implemented with compare-and-swap primitive to a
wait-free one. They adapted it for queues also using fetch-and-add. They provide
their own memory reclamation scheme which adds little overhead, unlike previous
work.
