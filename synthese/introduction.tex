Fast concurrent data structures are necessary to exploit the power of multi-core
processors and multiprocessors machine. To deliver high scalable performance,
applications may need shared objects (data structures) whose operations are
scalable. Each operations of an object (adding, removing, modifying an element)
much perform \textit{reasonably} well even when the number of threads sharing an
object is high.

One way to construct concurrent data structures is by using an object with
blocking operations. An example of such object would be a concurrent queue whose
operations have been protected with mutual exclusion (locks). This approach is
not scalable and it will lead to poor performance under high contention.
Unexpected delay from one thread prevents the others to use the object. The
source of the delay is multiple : operating system preemption, cache miss, page
fault, etc. Therefore designing non-blocking objects is the topic of lots of
past and ongoing researches.

We define here levels of process guarantees on each operation of a non-blocking
object \cite{Yang:2016:WQF:3016078.2851168}. Those are meant to predict how an
object will behave under high contention. The strongest guaranty is wait-free,
which means an operation will complete in a finite number of steps regardless of
how many threads is accessing this object and at what speed. Lock-free
operations complete in a finite number of steps only for \textit{some} threads
while it may not be the case for the others. Obstruction-free is the weakest
guaranty. Only one thread is guaranteed to complete an operation in a finite
number of steps.

While an object can be wait-free, this doesn't mean it will enable the
application to achieve high scalability. Actually, (*reference*) most past
wait-free queue implementations perform worse than others lock-free or blocking
queue. To achieve a wait-free guarantee, most wait-free implementations
sacrifice parallelism.

Having multiple threads concurrently modifying a shared object must not leave
this object (a queue for example) in an \textit{inconsistent} state. To ensure
the designed object is correct, we need a correctness condition for shared
objects. The authors choose linearizability for their design of concurrent queue.
Each operation must appear to \textit{take effect} instantaneously from the
point of view of other threads. The order of non-concurrent operations must also
be preserved, regardless of when the operation started and ended
\cite{Herlihy:1990:LCC:78969.78972}. An atomic operation or an operation
protected by mutual exclusion is by definition linearizable. When the operation
is not atomic, one must consider the point in time when changes made to the
shared object become visible to the other threads.

The use of atomic primitives such as compare-and-swap is critical in designing
Lock-free algorithm. To understand why, we must consider the consensus problem.
A consensus is reached between $n$ threads if those $n$ threads can
\textit{decide} on a value : each thread proposes one value to the others and
then chooses one value. The consensus is reached if all threads decides the same
value (consistency requirement) and if the common decision is one of the
initially proposed value (validity requirement).

Compare-and-swap primitive has an infinite consensus number
\cite{Herlihy:1991:WS:114005.102808} : it can solve the consensus problem for an
infinite number of threads. Compare-and-swap holds another important property
which is universality. An object is universal for $n$ threads if and only if it
has a consensus number greater or equal to $n$. It is possible to implement any
concurrent object in a wait-free manner if and only if the operation system
provides an universal object. Thus, compare-and-swap can be used to implement
any arbitrary concurrent wait-free object for any number of threads.

Unfortunately, the use of compare-and-swap on a variable shared by several
threads can drastically reduce performance even under low contention
(compare-and-swap retry problem \cite{Morrison:2013:FCQ:2517327.2442527}). In
this paper, C. Yang and J. Mellor-Crummey expose a new queue that is wait-free,
linearizable, and delivers high performance. They avoid common issues with past
designs such as the compare-and-swap problem by also relying on a fetch-and-add
primitive. They designed their wait-free queue using a methodology called
fast-path-slow-path \cite{Kogan:2012:MCF:2370036.2145835}. It can transform any
(?) non-blocking queue implemented with compare-and-swap to a wait-free one.
They adapted it for queues also using fetch-and-add.
