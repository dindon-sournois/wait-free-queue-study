\para{MS-Queue} The MS-Queue from Michael M. M. and Scott M. L. is a classical
simple lock-free queue implemented with only compare-and-swap as primitive
\cite{Michael96simple}. Their queue is designed as a singly-linked list, with
two references to the head and the tail of the queue. Their use compare-and-swap
to add (or remove) an element to the queue and to move the head and tail
reference. As a consequence, it is subject to the compare-and-swap retry
problem. Even with a low number of threads, performance are drastically reduced
under contention because of high probability of compare-and-swap failures.

The MS-Queue avoids the ABA problem by using double width compare-and-swap. Each
compare-and-swap to an address also increments a counter associated with that
address so that the comparison is done with one address and an integer
\cite{Herlihy08} \cite{Michael96simple}. \\

\para{Practical wait-free queue} One of the first implementation of a wait-free
queue supporting multiples enqueuers and dequeuers was designed by Kogan A. and
Petrank E. \cite{Kogan:2011:WQM:2038037.1941585}. This queue is based on the
MS-Queue. Each operation is divided into three atomic steps, threads can help
each others to complete a step without letting the possibility for steps
interleaving among the same type of operation (enqueue or dequeue). Because of
the overhead of this helping mechanism, it doesn't perform as well as the
MS-Queue. \\

\para{Combining-based queue} The P-Sim queue (wait-free) and CC-Queue (blocking)
from Fatourou P. and Kallimanis N. D. uses \textit{operation combining} to try
to achieve better scalability than compare-and-swap-based queue by reducing
synchronisation cost \cite{Fatourou:2011:HWU:1989493.1989549}
\cite{Fatourou:2012:RCS:2370036.2145849}. To do so, threads don't directly
modify the queue but publish a request. A single thread browses a list of
pending operations and applies them serially on the queue. While the CC-Queue
performs better than the P-Sim queue, it is still a blocking object and, also
because of the lack of parallelism, the CC-queue doesn't scale well. \\

\para{$FAA$-based LCRQ} The LCRQ is a lock-free queue designed by Morrison A.
and Afek Y. \cite{Morrison:2013:FCQ:2517327.2442527}. It is implemented as a
circular ring queue. Threads use fetch-and-add to get a cell index on the queue,
enqueue and dequeue are then done with double width compare-and-swap, which is
not universally available. However, by also relying on fetch-and-add, this queue
avoids the compare-and-swap retry problem. \\

\para{Fast-path-slow-path} The fast-path-slow-path methodology objective is to
construct a wait-free algorithm from a lock-free one
\cite{Kogan:2012:MCF:2370036.2145835}. This methodology relies on a
compared-and-swap-based lock-free algorithm \textit{most} of the time
(fast-path), and switch to a wait-free algorithm when too many compared-and-swap
failures are encountered by using an helping mechanism (slow-path). It aims to
achieve lock-free performance with a wait-free guaranty. They use the lock-free
MS-queue as the fast path.
