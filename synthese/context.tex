\para{MS-Queue} The MS-Queue from M. M. Michael and M. L. Scott is a classical
simple lock-free queue implemented with only compare-and-swap as a
synchronization primitive \cite{Michael96simple}. Their queue is designed as a
singly-linked list, with two references to the head and the tail of the queue.
Their use compare-and-swap to add (or remove) an element to the queue and to
move the head and tail reference. As a consequence, it is subject to the
compare-and-swap retry problem. Even with a low number of threads, performance
is drastically reduced under contention because of high probability of
compare-and-swap failures.

The MS-Queue avoids the ABA problem by using a compare-and-swap with
modification counters. Each compare-and-swap on an address also increments a
counter associated with that address so that two identical addresses can be
distinguished by how much they have been modified \cite{Herlihy08}
\cite{Michael96simple}. \\

\para{Practical wait-free queue} One of the first implementation of a wait-free
queue supporting multiples enqueuers and dequeuers was designed by A. Kogan and
E. Petrank \cite{Kogan:2011:WQM:2038037.1941585}. This queue is based on the
MS-Queue. Each operation is divided into three atomic steps. Threads can help
one another to complete a step without letting the possibility for steps
interleaving among the same type of operation (enqueue or dequeue). Because of
the overhead of this helping mechanism, it doesn't perform as well as the
MS-Queue. \\

\para{Combining-based queues} The P-Sim queue (wait-free) and CC-Queue (blocking)
from P. Fatourou and N. D. Kallimanis uses \textit{operation combining} to try
to achieve better scalability than compare-and-swap-based queue by reducing
synchronization cost \cite{Fatourou:2011:HWU:1989493.1989549}
\cite{Fatourou:2012:RCS:2370036.2145849}.

To do so, threads don't directly modify the queue but publish a request. A
single thread browses a list of pending operations and applies them serially on
the queue. While the CC-Queue performs better than the P-Sim queue, it is still
a blocking object and, also because of the lack of parallelism, the CC-queue
doesn't scale well. \\

\para{$FAA$-based LCRQ} The LCRQ is a lock-free queue designed by A. Morrison
and Y. Afek \cite{Morrison:2013:FCQ:2517327.2442527}. It is implemented as a
circular ring queue. Threads use fetch-and-add to get a cell index on the queue,
enqueue and dequeue are then done with double width compare-and-swap, which is
not universally available. However, by also relying on fetch-and-add, this queue
avoids the compare-and-swap retry problem. \\

\para{Fast-path-slow-path} Designed by A. Kogan and E. Petrank, the objective of
the fast-path-slow-path methodology is to construct a wait-free algorithm from a
lock-free one \cite{Kogan:2012:MCF:2370036.2145835}. The resulting algorithm
relies on a compare-and-swap-based lock-free algorithm \textit{most} of the time
(fast-path), and switch to a wait-free algorithm if too many compared-and-swap
failures occurred by using a helping mechanism (slow-path). It aims to achieve
lock-free performance with a wait-free guarantee. They experimented it with the
lock-free MS-queue as the fast path.
